\documentclass{article}
% Comment the following line to NOT allow the usage of umlauts
\usepackage[utf8]{inputenc}
% Uncomment the following line to allow the usage of graphics (.png, .jpg)
%\usepackage{graphicx}

% Start the document
\begin{document}

% Create a new 1st level heading
\section{EMERGING TECHNOLOGIES IN HEALTHCARE}

Devices, systems, and procedures, as well as vaccinations and drugs, are examples of health technologies that aid in the delivery of high-quality treatment, as well as lowering hospital and patient expenses and streamlining operations.Any programme or IT technology that boosts administrative productivity, streamlines workflow, and improves quality of life qualifies.


Supportive, instructional, information, organisational, rehabilitative, therapeutic, preventive, and diagnostic solutions are among the new technologies in healthcare that increase patient access and healthcare professional capacities.

\subsection{VIRTUAL CONCIERGE}

From answering emails and phone calls to scheduling appointments and activities, to responding to complaints and requests, and even collecting payments, a virtual concierge can do it all. Virtual concierge has a wide range of applications, and our front door solution is an outstanding illustration of one of them.


Their flagship product is an artificial intelligence-powered care navigation assistant that assists people in finding a clinic or physician, answers inquiries, makes appointments, and even tests for COVID-19 symptoms. Higher remote access capabilities, improved patient happiness, and better healthcare utilisation are all advantages for healthcare facilities.


These features, such as advanced analytics, real-time communication, and task management, provide flexibility and transparency, improve the patient experience, and reduce operational complexity. Healthcare facilities can use virtual concierge software to update their digital infrastructure and deliver customised solutions to patients.

\subsection{ARTIFICIAL INTELLIGENCE}

Artificial intelligence has the potential to improve healthcare by assisting professionals in making better decisions, reducing human error and the risk of avoidable scenarios. Advanced technology aids in the development of more efficient and accurate solutions, from imaging equipment and immunotherapy for cancer patients to identifying infectious disease patterns.


Learning algorithms are projected to have a substantial impact on healthcare services, including diagnostic techniques, treatments, and care processes, as they evolve and become more accurate.

\subsection{VOICE SEARCH}

According to Edison Search and NRP, approximately 17 percent of Americans now use a voice-activated speaker.
Patients' quest for care, including locating a hospital and inquiring about illness symptoms, is also altering as a result of smart technology assistants. According to the 2019 Voice Assistant Consumer Adoption in Healthcare research, 51.9 percent of customers are ready to use voice-search solutions, with 7.5 percent having done so already.


Assistants are primarily used by participants to inquire about symptoms (72.9 percent), obtain medical information (45.9 percent), locate an urgent care, clinic, or hospital (37.7 percent), and study treatment choices (37.7 percent).
Other applications include:

Purchasing pharmaceuticals


Locating a physical therapist or doctor


Setting up an appointment


Inquiring about the coverage of your insurance


Using a fitness tracker or other health device


Smart earbuds (6.5 percent), car-based assistants (12 percent), smart speakers (17.6 percent), and smartphones are the most popular voice-search devices among participants (52.6 percent).

\subsection{VIRTUAL REALITY}

Virtual reality (VR), which is also a new technology in healthcare, is a versatile tool that may be used to teach autistic children speech and social skills, as well as to engage patients in activities and games for rehabilitation. Through cognitive behavioural therapy and meditation training, virtual reality solutions can help people regulate hot flashes and decrease pain.


Some apps use Google Glass and augmented reality to help with clinical and medical documents, such as reminders, orders, and referrals.
VR solutions, which are built on HIPAA-compliant cloud infrastructure, assist healthcare providers in creating patient summaries and notes, responding to physician requests, pending orders, and writing referral letters.


Surgery centres, hospitals, emergency rooms, home visits, medical offices, urgent care clinics, pop-up clinics, and telemedicine are all places where virtual reality can be used.

\subsection{MOBILE APPS}

Mobile applications, which are a relatively new technology in healthcare, can upload patients' medical records, check in, plan appointments, and provide expert advice.
A variety of mobile apps now assist clinicians with patient monitoring and management, information gathering, consulting, and access to and maintenance of health records.
Medical workers can use apps to help with medical training and education, clinical decision-making, and time and data management, among other things.


They're useful for giving doctors communication capabilities including email, SMS, video conferencing, and voice calling.
Access to laboratory information systems, image archiving and communication systems, and electronic medical and health records is also required for physicians.
Clinical software programmes, such as medical calculators and illness diagnosis tools, are used by healthcare practitioners.


Mobile apps for patients include a variety of useful features, such as pill and care reminders, appointment management, physician selection, and remote medical help.
Patients can look for their hospital's location, consultation fees, visiting hours, and speciality while choosing a healthcare professional.
According to research published in PLoS One, waiting times, treatment continuity, communication skills, availability, residency/medical school, gender, and age all have a role in choosing a physician.


Many apps have a smart filter that allows users to select a doctor based on their degree, expertise, availability, and online reviews.
Patients can search for doctors and hospitals by name, as well as share information about their symptoms and ailments.

\section{EMERGING TECHNOLOGIES AS THE FUTURE OF HEALTHCARE}

Practices and hospitals are better equipped to diagnose and treat patients as they increasingly use mobile devices to access information ranging from medical records and history to research and pharmacological regimens.


In healthcare, new technology is critical for avoiding complications, avoiding unneeded surgeries, improving quality of life, and preserving health.
Innovations aid in the development of novel therapeutic approaches, treatment alternatives, and medications, as well as the prediction of epidemics' onset.


Emerging healthcare technology ranges from ankle braces and adhesive bandages to complex solutions like robotic prosthetic limbs and remote heart failure monitoring and diagnostic equipment, all of which contribute to improving patient outcomes and public health.
% Uncomment the following two lines if you want to have a bibliography
%\bibliographystyle{alpha}
%\bibliography{document}

\end{document}